\documentclass[oneside,11pt,a4paper,finnish]{book}

\usepackage[finnish,english]{babel}
\usepackage[utf8]{inputenc}
\usepackage{listings}
\usepackage[table]{xcolor}
\usepackage[parfill]{parskip}
\usepackage{tikz}
%\usepackage{tikz-qtree}
\usepackage{multicol}
\usepackage{hyperref}
\usepackage{array}
\usepackage{fouriernc}
\usepackage{graphicx}
\usepackage{framed}
\pagestyle{plain}

\date{\Large \today}

\usepackage[a4paper,vmargin=30mm,hmargin=38mm,footskip=15mm]{geometry}

%\titlespacing*{\chapter} {0pt}{0pt}{40pt}

\title{\Huge EppaBasic-käsikirja}
\author{\Large Antti Laaksonen}

\begin{document}

\selectlanguage{finnish}

\setcounter{page}{1}
\pagenumbering{roman}

\frontmatter
\maketitle
\setcounter{tocdepth}{0}
\tableofcontents

%\newpage
\setcounter{page}{1}
\pagenumbering{arabic}
\mainmatter

\newcommand{\kuva}[1] {
\begin{center}
\includegraphics[scale=0.4]{#1.png}
\end{center}
}

\chapter*{Johdanto}
\markboth{\MakeUppercase{Johdanto}}{}
\addcontentsline{toc}{chapter}{Johdanto}
lol


\chapter{Ensimmäinen ohjelma}

apina


\begin{verbatim}
apina
banaani
cembalo
\end{verbatim}

banaani

\chapter{Luvut ja merkkijonot}

Useimmat ohjelmat perustuvat pohjimmiltaan
lukujen (esim. $31$ ja $-5$) ja merkkijonojen (esim. "banaani") käsittelyyn.
Lukujen avulla voi laskea laskutoimituksia,
ja merkkijonojen avulla voi esittää tekstiä.
Tämä luku kertoo, kuinka lukuja ja merkkijonoja
käsitellään EppaBasicissa.

\section{Laskeminen}

EppaBasicia voi käyttää laskimena kirjoittamalla laskun
\texttt{Print}-komennon jälkeen.
Esimerkiksi seuraava ohjelma laskee neljä laskua:

\begin{verbatim}
Print 2 + 5
Print 55 - 209
Print 3 * (4 + 5)
Print 1 / 2
\end{verbatim}

Ohjelma tulostaa näytölle laskujen tulokset:

\begin{verbatim}
7
-154
27
0.5
\end{verbatim}

Tavalliset laskutoimitukset ovat \texttt{+} (yhteenlasku),
\texttt{-} (vähennyslasku), \texttt{*} (kertolasku)
sekä \texttt{/} (jakolasku).
Lisäksi merkki \texttt{\^} tarkoittaa potenssilaskua.
Sulkujen avulla voi muuttaa laskujärjestystä
kuten muutenkin matematiikassa.


Huomaa, että EppaBasicissa (ja kaikissa muissakin
ohjelmointikielissä) desimaaliluvuissa käytetään pistettä eikä pilkkua.

\newpage
\section{Merkkijono}

Merkkijono tarkoittaa suunnilleen sama kuin teksti.
Merkkijonon alussa on merkki \texttt{"},
tämän jälkeen tulee varsinainen teksti,
ja lopussa on vielä toinen merkki \texttt{"}.
Esimerkiksi \texttt{"abc"}, \texttt{"Uolevi"}
ja \texttt{"saippuakauppias"} ovat merkkijonoja.

Esimerkiksi ohjelma

\begin{verbatim}
Print "apina"
Print "banaani"
Print "cembalo"
\end{verbatim}

tulostaa seuraavat merkkijonot:

\begin{verbatim}
apina
banaani
cembalo
\end{verbatim}

Seuraava ohjelma taas laskee, montako sekuntia on vuodessa:

\begin{verbatim}
Print "Sekuntien määrä vuodessa:"
Print 365 * 24 * 60 * 60
\end{verbatim}

Ohjelman tulostus on seuraava:

\begin{verbatim}
Sekuntien määrä vuodessa:
31536000
\end{verbatim}

\section{Muuttuja}

Muuttujan ideana on, että siihen voi tallentaa
muistiin ohjelmassa tarvittavan tiedon.
Muuttujaan viitataan nimellä,
jonka tulee muodostua merkeistä \texttt{a}--\texttt{z}, \texttt{A}--\texttt{Z}
ja \texttt{0}--\texttt{9}. Nimen ensimmäinen merkki ei saa olla numero.

Seuraavassa ohjelmassa muuttujaan \texttt{nimi}
laitetaan ohjelman alussa päähenkilön nimi.
Tämän jälkeen nimeä käytetään monessa kohdassa tarinaa.

\begin{verbatim}
Dim nimi = "Uolevi"
Print "Tämän tarinan päähenkilö on " & nimi & "."
Print nimi & " on urhea ritari kaukaisessa maassa."
Print "Eräänä päivänä " & nimi & " nousi ratsulle varhain aamulla."
Print "Oli aika jättää kotikylä."
\end{verbatim}

Ohjelma tulostaa tarinan näin:

\begin{verbatim}
Tämän tarinan päähenkilö on Uolevi.
Uolevi on urhea ritari kaukaisessa maassa.
Eräänä päivänä Uolevi nousi ratsulle varhain aamulla.
Oli aika jättää kotikylä.
\end{verbatim}

Uuden muuttujan ottaa käyttöön komento \texttt{Dim},
jonka yhteydessä annetaan muuttujan nimi ja alkusisältö.
Tämän jälkeen muuttujaan voi viitata miten monta
kertaa tahansa myöhemmin ohjelmassa.
Muuttujan yhdistäminen tekstiin oikeisiin
kohtiin onnistuu käyttämällä merkkiä \texttt{\&}.

Tässä muuttujan hyötynä on,
että jos päähenkilön nimi vaihtuu,
muutos riittää päivittää yhteen paikkaan.
Esimerkiksi jos päähenkilö onkin Maija,
riittää muuttaa ohjelman aloitusriviksi

\begin{verbatim}
Dim nimi = "Maija"
\end{verbatim}

ja tarinasta tulee toisenlainen:

\begin{verbatim}
Tämän tarinan päähenkilö on Maija.
Maija on urhea ritari kaukaisessa maassa.
Eräänä päivänä Maija nousi ratsulle varhain aamulla.
Oli aika jättää kotikylä.
\end{verbatim}

\section{Monta muuttujaa}

Seuraavassa ohjelmassa on peräti kolme muuttujaa.
Muuttujassa \texttt{nimi} on päähenkilön nimi kuten ennenkin.
Tämän lisäksi muuttuja
\texttt{vuosi} sisältää tarinassa meneillään olevan vuoden
ja \texttt{ika} sisältää päähenkilön iän.

Huomaa, että muuttujan nimi ei voi sisältää kirjaimia
ä ja ö.
Niinpä nimi \texttt{ikä} ei kelpaa muuttujaksi,
vaan täytyy käyttää esimerkiksi nimeä \texttt{ika}.

\begin{verbatim}
Dim nimi = "Uolevi"
Dim vuosi = 1672
Dim ika = 23
Print "Tämän tarinan päähenkilö on " & nimi & "."
Print "Elämme vuotta " & vuosi & "."
Print nimi & " on iältään " & ika & "-vuotias."
Print "Hän syntyi siis vuonna " & vuosi - ika & "."
\end{verbatim}

Ohjelma tulostaa seuraavan tarinan:

\begin{verbatim}
Tämän tarinan päähenkilö on Uolevi.
Elämme vuotta 1672.
Uolevi on iältään 23-vuotias.
Hän syntyi siis vuonna 1649.
\end{verbatim}

Ohjelma yhdistää ensin uusia muuttujia tekstiin
samalla tavalla kuin ennenkin.
Viimeisellä rivillä ohjelma laskee muuttujista
\texttt{vuosi} ja \texttt{ika} päähenkilön syntymävuoden
vähentämällä muuttujasta \texttt{vuosi} muuttujan \texttt{ika}.

Jälleen tarinasta saa uudenlaisen vaihtamalla muuttujien sisällön.
Esimerkiksi jos muuttujat ovat

\begin{verbatim}
Dim nimi = "Maija"
Dim vuosi = 2050
Dim ika = 98
\end{verbatim}

tarinasta tulee

\begin{verbatim}
Tämän tarinan päähenkilö on Maija.
Elämme vuotta 2050.
Maija on iältään 98-vuotias.
Hän syntyi siis vuonna 1952.
\end{verbatim}

\section{Funktio}

Funktion tarkoituksena on
muuttaa jollain tavalla
sille annettua tietoa.
Sekä lukuihin että merkkijonoihin
liittyvät omat funktionsa.

Esimerkki lukuihin liittyvästä
funktiosta on \texttt{Sqr},
joka laskee luvun neliöjuuren.
Esimerkiksi luvun 16 neliöjuuri on 4,
joten ohjelma

\begin{verbatim}
Print Sqr(16)
\end{verbatim}

tuottaa seuraavan tuloksen:

\begin{verbatim}
4
\end{verbatim}

Seuraava ohjelma laskee neliöjuurella
neliön pinta-alasta sen sivun pituuden.

\begin{verbatim}
Dim ala = 123
Print "Jos neliön pinta-ala on " & ala & ","
Print "sen sivun pituus on " & Sqr(ala) & "."
\end{verbatim}

Ohjelman tulostus on seuraava:

\begin{verbatim}
Jos neliön pinta-ala on 123,
sen sivun pituus on 11.09053651.
\end{verbatim}

Seuraava ohjelma esittelee joitakin merkkijonofunktioita:

\begin{verbatim}
Dim nimi = "Uolevi"
Print "Nimen pituus on " & Len(nimi) & " merkkiä."
Print "Nimen alkuosa on " & Left(nimi, 3) & "."
Print "Nimen loppuosa on " & Right(nimi, 3) & "."
Print "Väärinpäin nimi on " & Reverse(nimi) & "."
\end{verbatim}

Ohjelman tulostus on seuraava:

\begin{verbatim}
Nimen pituus on 6 merkkiä.
Nimen alkuosa on Uol.
Nimen loppuosa on evi.
Väärinpäin nimi on iveloU.
\end{verbatim}

Funktio \texttt{Len} laskee merkkijonon pituuden.
Funktiot \texttt{Left} ja \texttt{Right}
ottavat annetun määrän merkkejä
merkkijonon alusta ja lopusta.
Funktio \texttt{Reverse} taas kääntää merkkijonon väärinpäin.

Löydät paljon lisää funktioita tutkimalla EppaBasicin komentolistaa.

\chapter{Piirtäminen}

\chapter{Ikkunat ja ehdot}

\chapter{Toistaminen}

\chapter{Animaatio}

\chapter{Taulukko ja arvonta}

\chapter{Pelin tekeminen}

\kuva{mato}

\chapter{Koodin osittelu}

\chapter{Algoritmit}

\end{document}